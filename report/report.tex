\documentclass{article}

\title{
    Project Report
    \\ \Large
    Methods and Models for Combinatorial Optimization 
    }
\author{Elia Scandaletti - 2087934}
% TODO: fix date
% \date{...}

\begin{document}
\maketitle

\section{Introduction}

This project consists in the development and comparison of two methods to solve the Traveling Salesman Problem in a specific domain, namely the drilling of holes in electric panel boards.
In particular, the problem consists in minimizing the time the drill needs to prepare each board.
Since the time it takes to drill a single hole is assumed to be constant, the problem is equivalent to minimizing the time needed to move the drill through each hole position.

The first method is based on a mixed integer linear programming formulation of the problem and always yield the optimal solution.
The second approach, instead, uses an heuristic method which may result in suboptimal solutions but should require a shorter time.

The first method uses the CPLEX library, whereas the second one leverages tabu search.
Both of them are implemented using C++.

\paragraph{Domain Characteristics and Assumptions}
The domain of the problem being restricted to the movement of a drill on electric panel boards allows to assume some realistic characteristics of the problem:
\begin{itemize}
    \item holes are aligned in a grid, likely in rectangular shapes;
    \item the number of holes on each board is between 10 and 250.
\end{itemize}
A further assumption is that the drill can move at constant speed in any direction.
We therefore assume that the time needed to move the drill from a hole to the next one is proportional to the linear distance between the two.

\section{Exact Method}

\subsection{MILP Formulation}

\subsection{Implementation}

\section{Heuristic Method}

\subsection{Algorithm Design}
\paragraph{Opt2 moves}
\paragraph{Opt2.5 moves}
\paragraph{Tabu List} (store opt2 moves)
\paragraph{Aspiration Criteria}
\paragraph{Non-improving iterations}
\paragraph{Stopping Criteria}

\subsection{Implementation}

\section{Experimental Results}

\subsection{Parameters Tuning}
\paragraph{Tabu List Length} vs size X 
\paragraph{Max Non Imp Iter}

\subsection{Results Comparison}


\end{document}