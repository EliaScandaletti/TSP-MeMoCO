\documentclass{article}

\usepackage{amsmath}
\usepackage{amssymb}

\title{
    Project Report
    \\ \Large
    Methods and Models for Combinatorial Optimization 
    }
\author{Elia Scandaletti - 2087934}
% TODO: fix date
% \date{...}

\begin{document}
\maketitle

\section{Introduction}

This project consists in the development and comparison of two methods to solve the Traveling Salesman Problem in a specific domain, namely the drilling of holes in electric panel boards.
In particular, the problem consists in minimizing the time the drill needs to prepare each board.
Since the time it takes to drill a single hole is assumed to be constant, the problem is equivalent to minimizing the time needed to move the drill through each hole position.

The first method is based on a mixed integer linear programming formulation of the problem and always yield an optimal solution.
The second approach, instead, uses an heuristic method which may result in suboptimal solutions.
On the other hand, it should require less computational resources, such as time, threads and memory.

The first method uses the CPLEX library, whereas the second one leverages tabu search.
Both of them are implemented using C++.

\paragraph{Notation}
In the rest of the report we will be using the following symbol with the meaning described below:
\begin{itemize}
    \item[$N$:] the nodes in the graph representation of the problem, i.e. the set of holes on the board.
    \item[$A$:] the arcs in the graph representation of the problem, i.e. the possible moves between two holes.
    \item[$c_{i j}$:] the cost of traveling over an arc, i.e. the time the drill takes to move from hole $i$ to hole $j$.
    \item[$0$:] an arbitrary node from which the traveler starts, i.e. the first hole to be drilled.
\end{itemize}

\paragraph{Domain Characteristics and Assumptions}
The domain of the problem being restricted to the movement of a drill on electric panel boards allows to assume some realistic characteristics of the problem:
\begin{itemize}
    \item holes are aligned in a grid, likely in rectangular shapes;
    \item the number of holes on each board is between 10 and 250.
\end{itemize}
A further assumption is that the drill can move at constant speed in any direction.
We therefore assume that the time needed to move the drill from a hole to the next one is proportional to the linear distance between the two.

\section{Exact Method}

\subsection{MILP Formulation}
The exact method uses the following mathematical formulation of the problem.
\begin{align}
    \min & \sum_{i,j : (i,j) \in A} c_{i j} y_{i j}                                                                                                 \\
    \label{MILP:consume}
    s.t. & \sum_{i : (i,k) \in A} x_{i k} - \sum_{j : (k,j) \in A, j \neq 0} x_{k j} &  & = 1                    &  & \forall k \in N               \\
    \label{MILP:output}
         & \sum_{j : (i,j) \in A} y_{i j}                                            &  & = 1                    &  & \forall i \in N               \\
    \label{MILP:input}
         & \sum_{i : (i,j) \in A} y_{i j}                                            &  & = 1                    &  & \forall j \in N               \\
    \label{MILP:activation}
         & x_{i j}                                                                   &  & \leq (|N| - 1) y_{i j} &  & \forall (i,j) \in A, j \neq 0 \\
         & x_{i j} \in \mathbb{R}                                                    &  &                        &  & \forall (i,j) \in A, j \neq 0 \\
         & y_{i j} \in \{0, 1\}                                                      &  &                        &  & \forall (i,j) \in A
\end{align}

The idea behind this formulation consider the TSP as a network flow problem.

There are two sets of variables. The real $x_{i j}$ variables indicate the amount of flow from node $i$ to node $j$, while the binary $y_{i j}$ variables indicate whether there is any flow between the noe $i$ and $j$.

The formulation assumes there is a flow of amount $(|N|-1)$ coming out of the starting node $0$.
Then it constraints each node to:
\begin{itemize}
    \item consume one unit of flow (\ref{MILP:consume});
    \item forward the flow only to one node (\ref{MILP:output});
    \item receive the flow only to one node (\ref{MILP:input}).
\end{itemize}

Finally, the constraint (\ref{MILP:activation}) ensures that if there is flow from node $i$ to node $j$, then $y_{i j} = 1$.
Th fact that if there is no flow from node $i$ to node $j$, then $y_{i j} = 0$ is ensured by optimality.

% This formulation corresponds to the one presented in \cite{gavish1978travelling}

\subsection{Implementation}

\section{Heuristic Method}

\subsection{Algorithm Design}
\paragraph{Opt2 moves}
\paragraph{Opt2.5 moves}
\paragraph{Tabu List} (store opt2 moves)
\paragraph{Aspiration Criteria}
\paragraph{Non-improving iterations}
\paragraph{Stopping Criteria}

\subsection{Implementation}

\section{Experimental Results}

\subsection{Parameters Tuning}
\paragraph{Tabu List Length} vs size X 
\paragraph{Max Non Imp Iter}

\subsection{Results Comparison}


\end{document}