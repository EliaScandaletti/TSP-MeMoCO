\documentclass{article}

\usepackage{amsmath}
\usepackage{amssymb}
\usepackage[
    backend=biber,
    sorting=ynt
]{biblatex}
\addbibresource{biblio.bib}

\title{
    Project Report
    \\ \Large
    Methods and Models for Combinatorial Optimization 
    }
\author{Elia Scandaletti - 2087934}
% TODO: fix date
% \date{...}

\begin{document}
\maketitle

\section{Introduction}

This project consists in the development and comparison of two methods to solve the Traveling Salesman Problem in a specific domain, namely the drilling of holes in electric panel boards.
In particular, the problem consists in minimizing the time the drill needs to prepare each board.
Since the time it takes to drill a single hole is assumed to be constant, the problem is equivalent to minimizing the time needed to move the drill through each hole position.

The first method is based on a mixed integer linear programming formulation of the problem and always yield an optimal solution.
The second approach, instead, uses an heuristic method which may result in suboptimal solutions.
On the other hand, it should require less computational resources, such as time, threads and memory.

The first method uses the CPLEX library, whereas the second one leverages tabu search.
Both of them are implemented using C++.

\paragraph{Notation}
In the rest of the report we will be using the following symbol with the meaning described below:
\begin{itemize}
    \item[$N$:] the nodes in the graph representation of the problem, i.e. the set of holes on the board.
    \item[$A$:] the arcs in the graph representation of the problem, i.e. the possible moves between two holes.
    \item[$c_{i j}$:] the cost of traveling over an arc, i.e. the time the drill takes to move from hole $i$ to hole $j$.
    \item[$0$:] an arbitrary node from which the traveler starts, i.e. the first hole to be drilled.
\end{itemize}

\paragraph{Domain Characteristics and Assumptions}
The domain of the problem being restricted to the movement of a drill on electric panel boards allows to assume some realistic characteristics of the problem:
\begin{itemize}
    \item holes are aligned in a grid, likely in rectangular shapes;
    \item the number of holes on each board is between 60 and 250.
\end{itemize}
A further assumption is that the drill can move at constant speed in any direction.
We therefore assume that the time needed to move the drill from a hole to the next one is proportional to the linear distance between the two.

\section{Exact Method}
\label{sec:exact}

\subsection{MILP Formulation}
The exact method uses the following mathematical formulation of the problem.
\begin{align}
    \min & \sum_{i,j : (i,j) \in A} c_{i j} y_{i j}                                                                                                 \\
    \label{MILP:consume}
    s.t. & \sum_{i : (i,k) \in A} x_{i k} - \sum_{j : (k,j) \in A, j \neq 0} x_{k j} &  & = 1                    &  & \forall k \in N               \\
    \label{MILP:output}
         & \sum_{j : (i,j) \in A} y_{i j}                                            &  & = 1                    &  & \forall i \in N               \\
    \label{MILP:input}
         & \sum_{i : (i,j) \in A} y_{i j}                                            &  & = 1                    &  & \forall j \in N               \\
    \label{MILP:activation}
         & x_{i j}                                                                   &  & \leq (|N| - 1) y_{i j} &  & \forall (i,j) \in A, j \neq 0 \\
         & x_{i j} \in \mathbb{R}                                                    &  &                        &  & \forall (i,j) \in A, j \neq 0 \\
         & y_{i j} \in \{0, 1\}                                                      &  &                        &  & \forall (i,j) \in A
\end{align}

The idea behind this formulation consider the TSP as a network flow problem.

There are two sets of variables.
The real $x_{i j}$ variables indicate the amount of flow from node $i$ to node $j$, while the binary $y_{i j}$ variables indicate whether there is any flow between the noe $i$ and $j$.

The formulation assumes there is a flow of amount $(|N|-1)$ coming out of the starting node $0$.
Then it constraints each node to:
\begin{itemize}
    \item consume one unit of flow (\ref{MILP:consume});
    \item forward the flow only to one node (\ref{MILP:output});
    \item receive the flow only to one node (\ref{MILP:input}).
\end{itemize}

Finally, the constraint (\ref{MILP:activation}) ensures that if there is flow from node $i$ to node $j$, then $y_{i j} = 1$.
Th fact that if there is no flow from node $i$ to node $j$, then $y_{i j} = 0$ is ensured by optimality.

This formulation corresponds to the one presented in \cite{gavish1978travelling}.

\section{Heuristic Method}
\label{sec:heuristic}

\subsection{Algorithm Design}
The chosen algorithm uses the tabu search meta-heuristic.
Each component of the algorithm will be discussed in the following paragraphs.

\paragraph{Initial Solution}
To generate the initial solution the farthest node insertion heuristic has been used.
This heuristic has been implemented in its deterministic flavour using the two farthest nodes as a starting loop.

\paragraph{2.5-opt Moves}
This type of move, described in \cite{johnson1997traveling}, is used to generate the neighborhood and consists in either a 2-opt move or a reposition move.
A 2-opt move consists in inverting a sub-tour, while a reposition move consists in moving a single node to a different position in the tour.
At each iteration there are $O(n^2)$ possible 2.5-opt moves.

\paragraph{Move Evaluation}
This type of move allows for incremental evaluation in constant time.
In particular, for a 2-opt move that invert the sub-tour $i$-$j$, we have
$$ \Delta C_{\textrm{2-opt}} = c_{p(i), j} + c_{i, s(j)} - c_{p(i), i} - c_{j, s(j)} $$
Instead, for a reposition move that shifts the node $i$ just after the node $j$, assuming that $j \neq i$ and $j$ is not immediately preceding $i$, we have
$$ \Delta C_{\textrm{rep}} = c_{p(i), s(i)} - c_{p(i), i} - c_{i, s(i)} - c_{j, s(j)} + c_{j, i} + c_{i, s(j)} $$

Where $p(i)$ is the node preceding $i$ in the current tour and $s(i)$ is the node following $i$ in the current tour.

\paragraph{Solution Representation}
The solution is represented as a sequence of nodes in order to easily apply 2.5-opt moves.
In the sequence each node appears exactly once.

In the solution, the traveler starts the tour from the first node and visits each node in the order they appear in the vector.
After visiting the last node, the traveler goes back to the first one.

\paragraph{Tabu List}
The tabu list used in the algorithm is fixed-size and keeps in memory the last 2.5-opt moves performed.
While choosing the next 2.5-opt move to perform, the algorithm excludes the ones in the list.

In some cases, though, the tabu list can be by-passed.
This happens when a move which is tabu leads to a solution which is the best found so far.

\paragraph{Exploring Strategy}
The algorithms performs a tabu search using the steepest descent criteria on the 2.5-opt neighborhood.
If the tabu search cannot find an improving solution for a given number of iterations, then the algorithm moves to the worst neighbor of the current solution and resume the tabu search from there.

\paragraph{Stopping Criteria}
The algorithms halts when one of the following conditions is met:
\begin{itemize}
    \item the maximum number of non increasing iterations is reached;
    \item the maximum number of iterations is reached.
\end{itemize}

\paragraph{Configuration of the Algorithm}
The following parameters of the algorithm that can be configured:
\begin{itemize}
    \item \textbf{Size of tabu list}
          This is the maximum number of moves stored in the tabu list.
          If this number is too high:
          \begin{itemize}
              \item it takes up more memory;
              \item it may exclude too many moves, including the ones that may lead to a (later) improvement.
          \end{itemize}
          On the other hand, if it is too low, the risk of moving around a local minimum without escaping from it increases.

          Furthermore, the tabu list helps not to fall again in the previous local minimum after the diversification steps. Therefore, a short tabu list is less effective in diversification and may lead to an overall worse solution.

    \item \textbf{Maximum number of non-improving tabu iterations}
          This parameter determines how many non-improving 2.5-opt moves can be done before performing a diversification step.
          If it is too high, the risk of moving around the same local minimum without finding a better solution for a long time is high.
          On the other hand, if it is too low, the algorithm could miss the opportunity to escape from a local minimum without resorting to differentiation.

    \item \textbf{Maximum number of non-improving iterations}
          This parameter determines how many non-improving moves, including both 2.5-opt moves and diversification steps, can be performed before the algorithm halts.
          The higher this value is, the higher the risk of performing useless iterations is, where useless means that they do not contribute to the overall solution.
          On the other hand, if it is too low, the algorithm could miss the opportunity of finding better solutions.

    \item \textbf{Maximum number of iterations}
          This is the maximum number of moves that the algorithm can perform.
          If it is too low, the algorithm may stop before reaching any good solution.
          If it is too high, the algorithm may run for a long time while yielding small or no improvement at all.

          The risk of a high maximum number of iterations may be mitigated by reducing the maximum number of non-improving iterations.
\end{itemize}

\section{Compilation and Execution}

\subsection{Compilation}
To compile the programs, it is sufficient to execute the command \texttt{make} while in the \texttt{src/} directory.
The compilation will result in two executable files in the same directory: \texttt{exact} and \texttt{tabu}.

\subsection{Execution}
The \texttt{exact} executable takes one argument which is the relative path of the input file.
The program apply the exact method described in Section \ref{sec:exact} to the input.

The \texttt{tabu} executable takes five arguments which are:
\begin{itemize}
    \item relative path of the input file;
    \item the size of the tabu list;
    \item the maximum number of non-improving tabu iterations;
    \item the maximum number of non-improving iterations;
    \item the maximum number of iterations.
\end{itemize}
The program apply the heuristic method described in Section \ref{sec:heuristic} to the input file, using the parameters specified in the arguments.

\subsection{Input format}
The input files are plain text files which contains an integer number and a matrix of floating point numbers.

The first number is interpreted as an integer and represent the size of the problem. Let's call it $n$.
The following $n^2$ numbers are interpreted as floating point numbers and represent the cost matrix of the problem.


Usually, for ease of reading, the first number is on its own line, while the matrix is written on $n$ lines, each containing $n$ numbers.
This is not enforced, though, since spaces, tabs and new lines are ignored.

\section{Implementation}

In this section, we first describe the classes used in both the methods, looking at their public interfaces.
Then we focus on the classes representing the solvers used in each method, diving into the details of the algorithms.

\subsection{Common Interfaces and Classes}
The following classes are abstract and represent common interfaces used in both the methods.

\paragraph{Instance}
This class represents a TSP instance.

It provides two virtual methods:
\begin{itemize}
    \item \texttt{n()}: returns the number of nodes in the instance;
    \item \texttt{cost(i, j)}: returns the cost of the arc from vertex $i$ to vertex $j$.
\end{itemize}

\paragraph{Solution}
This class represents a solution for a TSP problem.

It provides two virtual methods:
\begin{itemize}
    \item \texttt{length()}: returns the number of nodes in the solution;
    \item \texttt{evaluate(tsp)}: returns the cost of the solution when applied to the instance represented by \texttt{tsp}.
\end{itemize}

~\\
The following two classes implement the previous interfaces:

\paragraph{Matrix}
This class implements the \texttt{Instance} interface.
It stores the problem it represents as a matrix of edge costs.

This class has one constructor:
\begin{itemize}
    \item \texttt{Matrix(filename)}: reads the cost matrix from the file \texttt{filename}.
\end{itemize}

It adds to the interface the following method:
\begin{itemize}
    \item \texttt{read(filename)}: reads the cost matrix from the file \texttt{filename}.
\end{itemize}

\paragraph{Path}
This class implements the \texttt{Solution} interface.
It stores the problem it represents as a list of nodes, in the same order as they are visited by the traveler.

This class has three constructors:
\begin{itemize}
    \item \texttt{Path(tsp)}: constructs a standard solution which visits the node of the \texttt{tsp} instance using the order of their label;
    \item \texttt{Path(sol)}: default copy constructor;
    \item \texttt{Path(seq)}: construct the solution from the vector \texttt{seq}.
\end{itemize}

Within \texttt{Path}, three nested classes are declared:
\begin{itemize}
    \item \texttt{opt2}: represents a 2-opt move;
    \item \texttt{reposition}: represents a reposition move;
    \item \texttt{opt2\_5}: represents a 2.5-opt move.
\end{itemize}

Each of these three classes implement the equivalence operator.
Additionally, the \texttt{opt2\_5} class implements the overloaded static method \texttt{from} that converts \texttt{opt2} and \texttt{reposition} moves in \texttt{opt2\_5} moves.

\texttt{Path} also implements the following methods:
\begin{itemize}
    \item \texttt{get\_nth(n)}: returns the node in the $n$-th position of the tour;
    \item \texttt{evaluate\_move(tsp, m)}: incrementally evaluate the effect of applying move \texttt{m} on itself, when considering the instance \texttt{tsp};
    \item \texttt{apply\_move(m)}: apply move \texttt{m} on itself;
    \item \texttt{randomize()}: static method that returns a random \texttt{Path}.
\end{itemize}

The \texttt{evaluate\_move} and \texttt{apply\_move} are overloaded for all the three types of moves.

\subsection{MILP Formulation}
The exact method required one additional class \texttt{Flow}.

The class has the following fields:
\begin{itemize}
    \item \texttt{status}: is the CPLEX solver status;
    \item \texttt{env}: the CPLEX environment;
    \item \texttt{prob}: the CPLEX problem;
    \item \texttt{x\_idx}: a map from each arc to the column of the corresponding $x$ variable, such that $x_{i j}$ is stored in the column \texttt{x\_idx[i][j]} of \texttt{prob};
    \item \texttt{y\_idx}: the same as \texttt{x\_idx} except that it refers to the $y$ variables;
    \item \texttt{cost}: matrix that stores the cost of each edge, such that \texttt{cost[i][j]} corresponds to $c_{i j}$.
\end{itemize}

It has only one constructor, \texttt{Flow(tsp)}, which takes in input an instance and setup the fields described above.
This is done in three steps: firstly \texttt{cost}, \texttt{env} and \texttt{prob} are initialized, then the columns are generated one by one, lastly the rows are generated one by one.
At the end of the second step also \texttt{x\_idx} and \texttt{y\_idx} are setup.

After the problem is setup, it can be solved.
This is done by invoking the method \texttt{solve()} which firstly solve the problem using the CPLEX solver, then builds the corresponding \texttt{Path} by looking at the values of the $y$ variables.

The last method of the \texttt{Flow} class is \texttt{evaluate()}.
This methods return the value of the objective function, therefore it must be called after \texttt{solve()}.

\subsection{Tabu Search}
This method required two classes to be implemented.
The first one is the \texttt{FarthestInsertion} which implements a single method that solve the problem using the farthest insertion heuristic.
The second one is \texttt{Tabu}.

The class \texttt{Tabu} implements only the method \texttt{solve} which takes several parameters that configure the algorithm.
The parameters are:
\begin{itemize}
    \item \texttt{tsp}: the problem to be solved;
    \item \texttt{best\_sol}: the best solution known so far;
    \item \texttt{tabu\_size}: the size of tabu list;
    \item \texttt{max\_non\_imp\_local}: the maximum number of non-improving tabu iterations;
    \item \texttt{max\_non\_imp\_global}: the maximum number of non-improving iterations;
    \item \texttt{max\_iter}: the maximum number of iterations;
    \item \texttt{debug}: an optional output stream useful for debug, at every step the current state is output to this stream.
\end{itemize}

\section{Experimental Results}

\subsection{Parameters Tuning}
\paragraph{Tabu List Length} vs size X
\paragraph{Max Non Imp Iter}

\subsection{Results Comparison}

\nocite{*}
\printbibliography

\end{document}